\chapter{Performancetests der unterschiedlichen Programmiersprachen}
\label{cha:PerformancetestsderunterschiedlichenProgrammiersprachen}

In diesem Kapitel werden die Rahmenbedingungen der Tests beschrieben, die durchgeführten Performancetests analysiert und die Ergebnisse interpretiert.

\section{Die Testvoraussetzungen}
\label{sec:DieTestvoraussetzungen}

Damit die Ergebnisse der Performancetests miteinander vergleichen zu können, müssen die Tests unter den gleichen Voraussetzungen durchgeführt werden.
Deswegen wird in diesem Abschnitt beschrieben, welche Voraussetzungen für die Tests geschaffen wurden.

\subsection{Die Testumgebung}
\label{subsec:DieTestumgebung}

\subsection{Die Testdaten}
\label{subsec:DieTestdaten}

Alle Strategien, die in den Performancetests getestet werden, verwenden die gleichen historischen Kursdatensätze.
Für die Tests werden Kursdaten von drei verschiedenen Kryptomünzen verwendet:
\begin{itemize}
    \item Bitcoin (BTC)
    \item Ethereum (ETH)
    \item Cardano (ADA)
\end{itemize}

Diese Kursdaten werden von der Kryptobörse Binance \cite{Binance} bezogen und umfassen jeweils einen Zeitraum von ein, zwei und drei Jahren.
Für die Testfälle wird Bitcoin im Zeitraum vom 01.07.2023 bis zum 30.06.2025, Cardano im Zeitraum vom 01.01.2021 bis zum 31.12.2022 und 
Ethereum im Zeitraum vom 01.01.2023 bis zum 31.12.2023 betrachtet.
Die Kursdaten umfassen etwa 500.000 Datenpunkte pro Jahr und liegen in einem Minutenintervall vor. 
Somit stehen für Bitcoin rund 1,5 Millionen, für Cardano etwa 1 Million und für Ethereum etwa 500.000 Kurswerte zur Verfügung.
Durch die Kombination der Beobachtungszeiträume und Kryptowährungen wird ein breites Spektrum an unterschiedlichen Kursbewegungen abgebildet, 
sodass verschiedene Marktphasen und stark variierende Datenpunktumfänge in den Tests berücksichtig werden.

\section{Die Testausführung}
\label{sec:DieTestausführung}

\section{Die Testergebnisse}
\label{sec:DieTestergebnisse}

\subsection{Ergebnisse der gleitenden Durchschnittsüberschneidungs Strategie}
\label{subsec:ErgebnissedergleitendenDurchschnittsüberschneidungsStrategie}

\subsection{Ergebnisse der mittelwertrückkehr Strategie}
\label{subsec:Ergebnissedermittelwertrückkehr Strategie}

\subsection{Ergebnisse der Ausbruchsstrategie}
\label{subsec:ErgebnissederAusbruchsstrategie}

\section{Analyse der Testergebnisse}
\label{sec:AnalysederTestergebnisse}